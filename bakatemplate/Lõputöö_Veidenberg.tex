% Institute of Computer Science thesis template
% authors: Sven Laur, Liina Kamm
% last change Tõnu Tamme 03.05.2019
%--
% Compilation instructions:
% 1. Choose main language on line 55-56 (English or Estonian)
% 2. Compile 1-3 times to get refences right
% pdflatex bachelors-thesis-template
% bibtex bachelors-thesis-template
%--
% Please use references like this:
% <text> <non-breaking-space> <cite/ref-command> <punctuation>
% This is an example~\cite{example}.

\documentclass[12pt]{article}

% A package for setting layout and margins for your thesis 
\usepackage[a4paper]{geometry}

%%=== A4 page setup ===
%\setlength{\paperwidth}{21.0cm} 
%\setlength{\paperheight}{29.7cm}
%\setlength{\textwidth}{16cm}
%\setlength{\textheight}{25cm}


% When you write in Estonian then you want to use text with right character set
% By default LaTeX does not know what to do with õäöu letters. You have to specify
% a correct input and font encoding. For that you have to Google the Web     
%
% For TexShop under MacOS X. The right lines are 
%\usepackage[applemac]{inputenc}
%\usepackage[T1]{fontenc} %Absolutely critical for *hyphenation* of words with non-ASCII letters.
%
% For Windows and Linux the right magic lines are   
% \usepackage[latin1]{inputenc}
% \usepackage[latin5]{inputenc}
%
\usepackage[utf8]{inputenc} %standard encoding since 2018 (can be commented out?)
\usepackage[T1]{fontenc} %Absolutely critical for *hyphenation* of words with non-ASCII letters.

% Typeset text in Times Roman instead of Computer Modern (EC)
\usepackage{times}

% Suggested packages:
\usepackage{microtype}  %towards typographic perfection...
\usepackage{inconsolata} %nicer font for code listings. (Use \ttfamily for lstinline bastype)


% Use package babel for English or Estonian 
% If you use Estonian make sure that Estonian hyphenation is installed 
% - hypen-estonian or eehyp packages
%
%===Choose the main language in thesis
%\usepackage[estonian, english]{babel} %the thesis is in English 
\usepackage[english, estonian]{babel} %the thesis is in Estonian


% Change Babel document elements 
\addto\captionsestonian{%
  \renewcommand{\refname}{Viidatud kirjandus}%
  \renewcommand{\appendixname}{Lisad}%
}


% If you have problems with Estonian keywords in the bibliography
%\usepackage{biblatex}
%\usepackage[backend=biber]{biblatex}
%\usepackage[style=alphabetic]{biblatex}
%% plain --> \usepackage[style=numeric]{biblatex}
%% abbrv --> \usepackage[style=numeric,firstinits=true]{biblatex}
%% unsrt --> \usepackage[style=numeric,sorting=none]{biblatex}
%% alpha --> \usepackage[style=alphabetic]{biblatex}
%\DefineBibliographyStrings{estonian}{and={ja}}
%\addbibresource{bachelor-thesis.bib}


% General packages for math in general, theorems and symbols 
% Read ftp://ftp.ams.org/ams/doc/amsmath/short-math-guide.pdf for further information
\usepackage{amsmath} 
\usepackage{amsthm}
\usepackage{amssymb}

% Optional calligraphic fonts    
% \usepackage[mathscr]{eucal}

% Print a dot instead of colon in table or figure captions
\usepackage[labelsep=period]{caption}

% Packages for building tables and tabulars 
\usepackage{array}
\usepackage{tabu}   % Wide lines in tables
\usepackage{xspace} % Non-eatable spaces in macros

% Including graphical images and setting the figure directory
\usepackage{graphicx}
\graphicspath{{figures/}}

% Packages for getting clickable links in PDF file
%\usepackage{hyperref}
\usepackage[hidelinks]{hyperref} %hide red (blue,green) boxes around links
\usepackage[all]{hypcap}


% Packages for defining colourful text together with some colours
\usepackage{color}
\usepackage{xcolor} 
%\definecolor{dkgreen}{rgb}{0,0.6,0}
%\definecolor{gray}{rgb}{0.5,0.5,0.5}
\definecolor{mauve}{rgb}{0.58,0,0.82}


% Standard package for drawing algorithms
% Since the thesis in article format we must define \chapter for
% the package algorithm2e (otherwise obscure errors occur) 
\let\chapter\section
\usepackage[ruled, vlined, linesnumbered]{algorithm2e}

% Fix a  set of keywords which you use inside algorithms
\SetKw{True}{true}
\SetKw{False}{false}
\SetKwData{typeInt}{Int}
\SetKwData{typeRat}{Rat}
\SetKwData{Defined}{Defined}
\SetKwFunction{parseStatement}{parseStatement}


% Nice todo notes
\usepackage{todonotes}

% comments and verbatim text (code)
\usepackage{verbatim}


% Proper way to create coloured code listings
\usepackage{listings}
\lstset{ 
  %language=python,                % the language of the code
  language=C++,
  basicstyle=\footnotesize,        % the size of the fonts that are used for the code
  %numbers=left,                   % where to put the line-numbers
  %numberstyle=\footnotesize,      % the size of the fonts that are used for the line-numbers
  numberstyle=\tiny\color{gray}, 
  stepnumber=1,                    % the step between two line-numbers. If it's 1, each line 
                                   % will be numbered
  numbersep=5pt,                   % how far the line-numbers are from the code
  backgroundcolor=\color{white},   % choose the background color. You must add \usepackage{color}
  showspaces=false,                % show spaces adding particular underscores
  showstringspaces=false,          % underline spaces within strings
  showtabs=false,                  % show tabs within strings adding particular underscores
  frame = lines,
  %frame=single,                   % adds a frame around the code
  rulecolor=\color{black},		   % if not set, the frame-color may be changed on line-breaks within 
                                   % not-black text (e.g. commens (green here))
  tabsize=2,                       % sets default tabsize to 2 spaces
  captionpos=b,                    % sets the caption-position to bottom
  breaklines=true,                 % sets automatic line breaking
  breakatwhitespace=false,         % sets if automatic breaks should only happen at whitespace
  %title=\lstname,                 % show the filename of files included with \lstinputlisting;
                                   % also try caption instead of title
  keywordstyle=\color{blue},       % keyword style
  commentstyle=\color{dkgreen},    % comment style
  stringstyle=\color{mauve},       % string literal style
  escapeinside={\%*}{*)},          % if you want to add a comment within your code
  morekeywords={*,game, fun}       % if you want to add more keywords to the set
}


% Obscure packages to write logic formulae and program semantics
% Unless you do a bachelor thesis on program semantics or static code analysis you do not need that
% http://logicmatters.net/resources/ndexamples/proofsty3.html <= writing type rules => use semantic::inference
% ftp://tug.ctan.org/tex-archive/macros/latex/contrib/semantic/semantic.pdf
\usepackage{proof}
\usepackage{semantic} 
\setlength{\inferLineSkip}{4pt}
\def\predicatebegin #1\predicateend{$\Gamma \vdash #1$}

% If you really want to draw figures in LaTeX use packages tikz or pstricks
% However, getting a corresponding illustrations is really painful  


% Define your favorite macros that you use inside the thesis 
% Name followed by non-removable space
\newcommand{\proveit}{ProveIt\xspace}

% Macros that make sure that the math mode is set
\newcommand{\typeF}[1] {\ensuremath{\mathsf{type_{#1}}}\xspace}
\newcommand{\opDiv}{\ensuremath{\backslash \mathsf{div}}\xspace} 

% Nice Todo box
\newcommand{\TODO}{\todo[inline]}

% A way to define theorems and lemmata
\newtheorem{theorem}{Theorem}



%%% BEGIN DOCUMENT
\begin{document}

%===BEGIN TITLE PAGE
\thispagestyle{empty}
\begin{center}

\iflanguage{english}{%
\large
UNIVERSITY OF TARTU\\%[2mm]
Institute of Computer Science\\
Computer Science Curriculum\\%[2mm]
}{%
TARTU ÜLIKOOL\\
Arvutiteaduse instituut\\
Informaatika õppekava\\%[2mm]
}%\iflanguage

%\vspace*{\stretch{5}}
\vspace{25mm}

\Large Pärt Veidenberg

\vspace{4mm}

\huge Tõlkerakendus mitmekeelsuse haldamiseks

%\vspace*{\stretch{7}}
\vspace{20mm}

\iflanguage{english}{%
\Large Bachelor's Thesis (9 ECTS)
}{%
\Large Bakalaureusetöö (9 EAP)
}%\iflanguage

\end{center}

\vspace{2mm}

\begin{flushright}
 {
 \setlength{\extrarowheight}{5pt}
 \begin{tabular}{r l} 
  \sffamily \iflanguage{english}{Supervisor}{Juhendaja}: & \sffamily Vambola Leping \\
 \end{tabular}
 }
\end{flushright}

%\vspace*{\stretch{3}}
%\vspace{10mm}

\vfill
\centerline{Tartu 2021}

%===END TITLE PAGE

% If the thesis is printed on both sides of the page then 
% the second page must be must be empty. Comment this out
% if you print only to one side of the page comment this out
%\newpage
%\thispagestyle{empty}    
%\phantom{Text to fill the page}
% END OF EXTRA PAGE WITHOUT NUMBER


%===COMPULSORY INFO PAGE
\newpage

%=== Info in English
\newcommand\EngInfo{{%
\selectlanguage{english}
\noindent\textbf{\large Translation application for multi-language support}

\vspace*{3ex}

\noindent\textbf{Abstract:}

\noindent
The aim of this Bachelor's thesis is to develop a web application for multi-language support. The thesis contains an analysis of popular localization applications, localization application itself, technologies used and process of developing a web application.  

\vspace*{1ex}

\noindent\textbf{Keywords:}\\
web application, translation keys, localization, JavaScript, React.js, Node.js, GraphQL

\vspace*{1ex}

\noindent\textbf{CERCS:} P170 Computer science, numerical analysis, systems, control

\vspace*{1ex}
}}%\newcommand\EngInfo


%=== Info in Estonian
\newcommand\EstInfo{{%
\selectlanguage{estonian}
\noindent\textbf{\large Tõlkerakendus mitmekeelsuse haldamiseks}
\vspace*{1ex}

\noindent\textbf{Lühikokkuvõte:} 

\noindent
Käesoleva bakalaureusetöö eesmärgiks on veebipõhise rakenduse loomine mitmekeelsuse haldamiseks. Töö sisaldab tuntumate lokaliseerimisrakenduste analüüsi, veebirakendust ja kasutatuid tehnoloogiaid ning rakenduse valmimise kirjeldust.
\vspace*{1ex}

\noindent\textbf{Võtmesõnad:}\\ veebirakendus, tõlkevõtmed,  lokaliseerimine, JavaScript,  React.js, Node.js, GraphQL

\vspace*{1ex}

\noindent\textbf{CERCS:}\\ P170 Arvutiteadus, arvutusmeetodid, süsteemid, juhtimine

\vspace*{1ex}
}}%\newcommand\EstInfo


%=== Determine the order of languages on Info page
\iflanguage{english}{\EngInfo}{\EstInfo}
\iflanguage{estonian}{\EngInfo}{\EstInfo}


\newpage
\tableofcontents

\newpage
\section{Sissejuhatus}

Bakalaureusetöö eesmärk on analüüsida olemasolevaid lokaliseerimisrakendusi ja luua ise veebipõhine rakendus teiste veebi- ja mobiilirakenduste lokaliseerimiseks.
Lokaliseerimine on tõlgitava teksti kohandamine lõppkasutaja kultuuriruumi traditsioonide ja seadustega mis omakorda võimaldab laiendada enda pakutavaid teenuseid globaalselt.

Bakalaureusetöö on jaotatud neljaks peatükiks. Esimeses peatükis analüüsitakse olemasolevaid lok

\newpage
\section{Olemasolevate lokaliseerimisrakenduste analüüs} 

\subsection{Crowdin}

Some text...

\subsection{i18next} 

Some text

\subsection{Transifex}

Some text

\newpage
\section{Veebirakenduse ülesehitus}

\subsection{Kliendi pool}

Some text

\subsection{Serveri poool}

Some text

\subsection{Andmebaas}

Some text

\newpage
\section{Kasutatud tehnoloogiad}

\subsection{Node.js}

Some text

\subsection{PostgresQL}

Some text

\subsection{GraphQL}

Some text

\subsection{TypeOrm}

Some text

\subsection{Reactl}

Some text

\subsection{TypeScript}

Some text

\newpage
\section{Arengu võimalused}

New features

\clearpage
\section{Kokkuvõte}

Some text

\newpage

% BibTeX bibliography
\bibliographystyle{alpha} %plain=[1], alpha=[BGZ09]
\bibliography{bachelor-thesis}

\addcontentsline{toc}{section}{\refname}


% Use Biblatex if you have problems with Estonian keywords
%\printbibliography %biblatex


% Use alternative local LaTeX bibliography
\begin{comment}
\begin{thebibliography}{9}
\bibitem{proVerif} 
  Bruno Blanchet. 
  Proverif: Cryptographic protocol verifier in the formal model.
  \url{http://www.proverif.ens.fr/}.
  (checked 15.05.2012)
\bibitem{GameB_1} GameB1
\bibitem{GameB_2} GameB2
\bibitem{certicrypt} certicrypt
\bibitem{kamm12} kamm12
\end{thebibliography}
\end{comment}


\newpage
%\appendix
%\section*{\appendixname}
\iflanguage{english}%
  {\section*{Appendix}
  \addcontentsline{toc}{section}{Appendix}
  }%
  {\section*{Lisad}
  \addcontentsline{toc}{section}{Lisad}}


\section*{I. Glossary}
\addcontentsline{toc}{subsection}{I. Glossary}

\newpage

%=== Licence in English
\newcommand{\licencehint}[2]{\\\hspace*{#1}\textsl(#2)\par}
\newcommand\EngLicence{{%
\selectlanguage{english}
\section*{II. Licence}

\addcontentsline{toc}{subsection}{II. Licence}

\subsection*{Non-exclusive licence to reproduce thesis and make thesis public}

I, \textbf{Alice Cooper}, %author's name
  \licencehint{10mm}{author's name}

\begin{enumerate}
\item
herewith grant the University of Tartu a free permit (non-exclusive licence) to
\par
reproduce, for the purpose of preservation, including for adding to the DSpace digital archives until the expiry of the term of copyright,
\par
\textbf{Type Inference for Fourth Order Logic Formulae}, %
  \licencehint{10mm}{title of thesis}
\par
supervised by Axel Rose and May Flower. %supervisor's name
  \licencehint{10mm}{supervisor's name}
\item
I grant the University of Tartu a permit to make the work specified in p. 1 available to the public via the web environment of the University of Tartu, including via the DSpace digital archives, under the Creative Commons licence CC BY NC ND 3.0, which allows, by giving appropriate credit to the author, to reproduce, distribute the work and communicate it to the public, and prohibits the creation of derivative works and any commercial use of the work until the expiry of the term of copyright.
\item
I am aware of the fact that the author retains the rights specified in p. 1 and 2.
\item
I certify that granting the non-exclusive licence does not infringe other persons' intellectual property rights or rights arising from the personal data protection legislation. 
\end{enumerate}

\noindent
Alice Cooper\\ %author's name
\textbf{\textsl{dd/mm/yyyy}}
}}%\newcommand\EngLicence


%=== Licence in Estonian
\newcommand\EstLicence{{%
\selectlanguage{estonian}
\section*{II. Litsents}

\addcontentsline{toc}{subsection}{II. Litsents}

\subsection*{Lihtlitsents lõputöö reprodutseerimiseks ja üldsusele kättesaadavaks tegemiseks}

Mina, \textbf{Pärt Veidenberg}, %author's name
  \licencehint{10mm}{autori nimi}

\begin{enumerate}
\item
annan Tartu Ülikoolile tasuta loa (lihtlitsentsi) minu loodud teose
\par
\textbf{Tõlkerakendus mitmekeelsuse haldamiseks}, %title of thesis
    \licencehint{10mm}{lõputöö pealkiri}
\par
mille juhendaja(d) on Vambola Leping, %supervisor's name(s)
  \licencehint{10mm}{juhendaja nimi}
\par
reprodutseerimiseks eesmärgiga seda säilitada, sealhulgas lisada digitaalarhiivi DSpace kuni autoriõiguse kehtivuse lõppemiseni.
\par
\item
Annan Tartu Ülikoolile loa teha punktis 1 nimetatud teos üldsusele kättesaadavaks Tartu Ülikooli veebikeskkonna, sealhulgas digitaalarhiivi DSpace kaudu Creative Commonsi litsentsiga CC BY NC ND 3.0, mis lubab autorile viidates teost reprodutseerida, levitada ja üldsusele suunata ning keelab luua tuletatud teost ja kasutada teost ärieesmärgil, kuni autoriõiguse kehtivuse lõppemiseni.
\item
Olen teadlik, et punktides 1 ja 2 nimetatud õigused jäävad alles ka autorile.
\item
Kinnitan, et lihtlitsentsi andmisega ei riku ma teiste isikute intellektuaalomandi ega isikuandmete kaitse õigusaktidest tulenevaid õigusi. 
\end{enumerate}

\noindent
Pärt Veidenberg\\ %author's name
\textbf{\textsl{08.02.2021}}
}}%\newcommand\EstLicence


%===Choose the licence in active language
\iflanguage{english}{\EngLicence}{\EstLicence}


\end{document}

